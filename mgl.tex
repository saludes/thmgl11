\documentclass[adraft,copyright,creativecommons]{eptcs}
\providecommand{\event}{THedu'11}
%\usepackage{breakurl} % Not needed if you use pdflatex only.
\usepackage{listings}
\usepackage{color}
\definecolor{light-gray}{gray}{0.95}

\title{The GF Mathematics Library}
\author{Jordi Saludes
\institute{UPC}
\institute{Sistemes Avan\c cats de Control}\\
\email{jordi.saludes@upc.edu}
\and
Sebastian Xamb\'o
\institute{Universitat Polit\`ecnica de Catalunya.\\Barcelona, Spain}
\email{\quad ??}
}
\def\titlerunning{GF Math Library}
\def\authorrunning{J. Saludes \& S. Xamb\'o}
\begin{document}
\lstset{backgroundcolor=\color{light-gray},
		breaklines=true,
		tabsize=2} % language=GF
\maketitle

\begin{abstract}
The \emph{Mathematics Grammar Library} (mgl for short)
\end{abstract}

\section{Introduction} % (fold)
\label{sec:introduction}

Using the \emph{Grammatical Framework\cite{GF}}.

?? Descriure la historia: Projecte webALT ??

\subsection{Goals} % (fold)
\label{sub:goals}
The objective of the library is to allow natural language services for mathematical constructs at the level of high school and
college freshmen linear algebra and calculus, that eventually could grow up to cover more sophisticated productions.

At the present stage, the visible goal is to provide rendering of simple mathematical exercises in multiple languages.
A demo is available (see \cite{MathBar})

% subsection goals (end)

% section introduction (end)

\section{The library} % (fold)
\label{sec:the_library}
The library is a collection of GF modules

\subsection{Categories} % (fold)
\label{sub:categories}

GF types are called \emph{categories}, they are used to declare signatures for the functions that generate the grammar.
Ours are all possible combinations of \textbf{Variable} and \textbf{Value} with the mathematical categories: \textbf{Number}, \textbf{Set}, \textbf{Tensor} and \textbf{Function}.
Thus the category \texttt{VarNum} denotes a numeric variable like $x$, whilst \texttt{ValSet} denotes an actual set like ``the domain of the natural logarithm''.
The distinction between variables and values allows us to type-check productions like lambda abstractions that require a variable as the first argument.
Obviously variables can be promoted to values when needed.

Other categories stand for propositions, geometric constructions and indices.
% subsection categories (end)

\subsection{Structure} % (fold)
\label{sub:structure}

The library is organised matrix-like with an horizontal axis ranging over the targeted natural languages.
At the moment these are: Bulgarian, Catalan, English, Finnish, French, German, Italian, Romanian, Spanish and Swedish.

The vertical axis is for complexity and contains from bottom to top, three layers:

\begin{enumerate}
	\item The \emph{Ground} layer;
	\item The \emph{OpenMath} layer;
	\item The \emph{Operations} layer.
\end{enumerate}

The ground layer is very shallow and stands for literals, indices and variables.
The intermediate layer is called \emph{OpenMath} because is modelled after \emph{OpenMath Content Dictionaries} which are sets of mathematical objects collected by the \emph{OpenMath consortium\cite{OpenMath}}.

Each \emph{Content Dictionary} corresponds to a module in this layer.

The top layer (\emph{Operations}) deals with simple mathematical exercises like compute, find, prove, give an example \ldots.

% subsection structure (end)

\subsection{Examples} % (fold)
\label{sub:examples}

An example from the \emph{OpenMath} layer:


\begin{lstlisting}
mkProp
	(lt_num
		(abs (plus (BaseValNum (Var2Num x) (Var2Num y))))
		(plus (BaseValNum (abs (Var2Num x)) (abs (Var2Num y)))))
\end{lstlisting}
gives in Spanish:
\begin{quote}
El valor absoluto de la suma  de x y de y es menor que la suma del valor absoluto  de x y del valor absoluto de y
\end{quote}


\begin{lstlisting}
DoSelectFromN
	(Var2Num y)
	(domain (inverse tanh))
	(mkProp 
		(gt_num 
			(At cosh (Var2Num y))
			pi))	
\end{lstlisting}
gives in English:
\begin{quote}
Select y from the domain of the inverse of the hyperbolic tangent such that
the hyperbolic cosine of y is greater than pi.
\end{quote}




% subsection examples (end)

% section the_library (end)


\nocite{*}
\bibliographystyle{eptcsstyle/eptcs}
\bibliography{webalt}
% section Bibliography (end)
\end{document}
