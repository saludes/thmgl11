\documentclass[adraft,copyright,creativecommons]{eptcs}
\providecommand{\event}{THedu'11}
%\usepackage{breakurl}             % Not needed if you use pdflatex only.
\usepackage{listings}

\title{The GF Mathematics library}
\author{Jordi Saludes
\institute{UPC}
\institute{Sistemes Avan\c cats de Control}\\
\email{jordi.saludes@upc.edu}
\and
Sebastian Xamb\'o
\institute{Universitat Polit\`ecnica de Catalunya.\\Barcelona, Spain}
\email{\quad ??}
}
\def\titlerunning{GF math Library}
\def\authorrunning{J. Saludes \& S. Xamb\'o}
\begin{document}
\lstset{language=GF}
\maketitle

\begin{abstract}
The \emph{Mathematics Grammar Library} (mgl for short)
\end{abstract}

\section{Introduction} % (fold)
\label{sec:introduction}


?? Descriure la historia: Projecte webALT ??

\subsection{Goals} % (fold)
?? Afegir ??
\label{sub:goals}

% subsection goals (end)

% section introduction (end)

\section{The library} % (fold)
\label{sec:the_library}

\subsection{Categories} % (fold)
\label{sub:categories}

GF types are called \emph{categories}, they are used to declare signatures for the functions that generate the grammar.
Ours are all possible combinations of \textbf{Variable} and \textbf{Value} with the mathematical categories: \textbf{Number}, \textbf{Set}, \textbf{Tensor} and \textbf{Function}.
Thus the category \texttt{VarNum} denotes a numeric variable like $x$, whilst \texttt{ValSet} denotes an actual set like ``the domain of the natural logarithm''.
The distinction between variables and values allows us to typecheck productions like a lambda abstraction that requires a variable as the first argument.
Obviously variables can be promoted to values when needed.

Other categories stand for \emph{propositions}, geometric constructions and indices.
% subsection categories (end)

\subsection{Structure} % (fold)
\label{sub:structure}

The library is organized matrix-like with an horizontal axis ranging over the targeted natural languages.
At the moment these are: Bulgarian, Catalan, English, Finnish, French, German, Italian, Romanian, Spanish and Swedish.

The vertical axis is for complexity and contains from bottom to top, three layers:

\begin{enumerate}
	\item The \emph{Ground} layer;
	\item The \emph{OpenMath} layer;
	\item The \emph{Operations} layer.
\end{enumerate}

The ground layer is very shallow and stands for literals, indices and variables.
The intermediate layer is called \emph{OpenMath} because is modeled after \emph{OpenMath Content Dictionaries} which are sets of mathematical objects collected by the \emph{OpenMath consortium}. 

Each \emph{Content Dictionary} corresponds to a module in this layer.

% subsection structure (end)

\subsection{Examples} % (fold)
\label{sub:examples}

An example from the \emph{OpenMath} layer:


\begin{lstlisting}
mkProp (
	lt_num (
		abs (plus (
				BaseValNum (Var2Num x) (Var2Num y))))
		(plus (
			BaseValNum (abs (Var2Num x)) (abs (Var2Num y)))))
\end{lstlisting}
gives in Spanish:
\begin{quote}
El valor absoluto de la suma  de x y de y es menor que la suma del valor absoluto  de x y del valor absoluto de y
\end{quote}


\begin{lstlisting}
DoSelectFromN
	(Var2Num y)
	(domain (fns1_inverse tanh))
	(mkProp 
		(gt_num 
			(At cosh (Var2Num y))
			nums1_pi))	
\end{lstlisting}
gives in English:
\begin{quote}
Select y from the domain of the inverse of the hyperbolic tangent such that
the hyperbolic cosine of y is greater than pi.
\end{quote}




% subsection examples (end)

% section the_library (end)




\section{Bibliography} % (fold)

Often an official publication is only available against payment, but
as a courtesy to readers that do not wish to pay, the authors also
make the paper available free of charge at a repository such as
\url{arXiv.org}. In such a case it is recommended to also refer and
link to the URL of the response page of the paper in such a
repository.  This can be done using the bibtex fields {\tt ee} or {\tt
url}, which are treated as synonyms.  These fields should not be used
to duplicate information that is already provided through the DOI of
the paper.
You can find archival-quality URL's for most recently published papers
in DBLP---they are in the bibtex-field {\tt ee}. In fact, it is often
useful to check your references against DBLP records anyway, or just find
them there in the first place.


\nocite{*}
\bibliographystyle{eptcsstyle/eptcs}
\bibliography{webalt}
% section Bibliography (end)
\end{document}
