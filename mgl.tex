\documentclass[adraft,copyright,creativecommons]{eptcs}
\providecommand{\event}{THedu'11}
%\usepackage{breakurl}             % Not needed if you use pdflatex only.

\title{The GF Mathematics library}
\author{Jordi Saludes
\institute{UPC}
\institute{Sistemes Avan\c cats de Control}\\
\email{jordi.saludes@upc.edu}
\and
Sebastian Xamb\'o
\institute{Universitat Polit\`ecnica de Catalunya.\\Barcelona, Spain}
\email{\quad ??}
}
\def\titlerunning{GF math Library}
\def\authorrunning{J. Saludes \& S. Xamb\'o}
\begin{document}
\maketitle

\begin{abstract}
This is a sentence in the abstract.
This is another sentence in the abstract.
This is yet another sentence in the abstract.
This is the final sentence in the abstract.
\end{abstract}

\section{Introduction}




\section{Bibliography}

Often an official publication is only available against payment, but
as a courtesy to readers that do not wish to pay, the authors also
make the paper available free of charge at a repository such as
\url{arXiv.org}. In such a case it is recommended to also refer and
link to the URL of the response page of the paper in such a
repository.  This can be done using the bibtex fields {\tt ee} or {\tt
url}, which are treated as synonyms.  These fields should not be used
to duplicate information that is already provided through the DOI of
the paper.
You can find archival-quality URL's for most recently published papers
in DBLP---they are in the bibtex-field {\tt ee}. In fact, it is often
useful to check your references against DBLP records anyway, or just find
them there in the first place.


\nocite{*}
\bibliographystyle{eptcsstyle/eptcs}
\bibliography{webalt}
\end{document}
