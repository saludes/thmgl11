\documentclass[adraft,copyright,creativecommons]{eptcs}
\providecommand{\event}{THedu'11}
%\usepackage{breakurl}             % Not needed if you use pdflatex only.
\usepackage{listings}
\usepackage{color}
\definecolor{light-gray}{gray}{0.95}

\title{The GF Mathematics Library}
\author{Jordi Saludes
\institute{Universitat Polit\`ecnica de Catalunya}
\institute{Sistemes Avan\c cats de Control and MA2}
%Edifici GAIA, Terrassa (Spain)
\email{jordi.saludes@upc.edu}
\and
Sebastian Xamb\'o
\institute{Universitat Polit\`ecnica de Catalunya}
\institute{MA2, Edifici Omega, Barcelona (Spain)}
\email{\quad sebastia.xambo@upc.edu}
}
\def\titlerunning{GF Math Library}
\def\authorrunning{J. Saludes \& S. Xamb\'o}
\begin{document}
\lstset{backgroundcolor=\color{light-gray},
		breaklines=true,
		basicstyle=\ttfamily,
		% language=GF,
		tabsize=2}
\maketitle
\newcommand{\molto}{\textsc{mOlto}}
\newcommand{\webalt}{\textsc{WebALT}}
\newcommand{\openmath}{\textsc{OpenMath}}
\newcommand{\CD}{\textsc{CD}}
\newcommand{\MGL}{\textsc{MGL}}
\newcommand{\MMA}{\textsc{MMA}}
\newcommand{\CAS}{\textsc{CAS}}
\newcommand{\GF}{\textsc{GF}}

\newcommand{\Nat}{\texttt{Nat}}
\newcommand{\Prop}{\texttt{Prop}}



\begin{abstract}
The aim of this paper is to present
the \emph{Mathematics Grammar Library}.
The main points are
the context in which it originated,
its current design and functionality
and the present development goals.
We also comment on possible future applications
in the area of artificial mathematics assistants.
\end{abstract}





\section{Introduction}

An archetypal meeting point for natural language processing and mathematics
education is the realm of \emph{word problems}
(\cite{wikipedia-wordproblem, Verschaffel-Greer-DeCorte-2000}), a realm in
which mechanised mathematics assistants (\MMA) are expected to play an ever
more prominent role in the years to come.  The Mathematics Grammar Library
(\MGL) presented in this paper is not an \MMA, but our working hypothesis
is that it is a good basis on which to build useful \MMA's for learning and
teaching (cf. \cite{E-LearningMathematics, AutonomousLearners} for some
general clues on e-learning technologies).

In fact, we regard \MGL\ as a technology for 
providing multilinguality to
dialog systems capable of helping students in learning how to solve word
problems.  This assessment is based on the \MGL\ potential capabilities
for dealing effectively with a mixture of text and mathematical
expressions, and also for managing interactions with ancillary \CAS{}
systems.

To be more specific, the current general aim for MGL is to provide natural
language services for mathematical constructs at the level of high school
and college freshmen linear algebra and calculus.  At the present stage,
the concrete goal is to provide rendering of simple mathematical exercises
in multiple languages (see \cite{MathBar} for a demo).



\section{Origin}

For a closer view of \MGL, let us look briefly at its origins.  The
idea behind \MGL\ was born, to a good extend, on reflecting about one of
the key results of the \webalt\ project (see
\cite{Caprotti_multilingualdelivery, Caprotti06webalt!deliver, Nganga06}).  In
summary, the unfolding of this reflection went as follows.

One of the aims of \webalt{} was to produce a proof-of-concept
platform for the creation of a multilingual repository of \emph{simple}
mathematical problems with guaranteed quality of the (machine)
translations, in both linguistic and mathematical terms. The languages
envisioned were Catalan,
English,
Finnish,
French,
Italian and
Spanish.
Of these, Finnish, with its great complexities, could not be raised to
the same level of functionality as the others.

The WebALT prototype was successful and, as far as we know, that endeavour
brought about the first large application of the GF system (\cite{GF,Ranta11})
for the multilingual translation of simple mathematical questions.  The
powerful \GF\ scheme, based on the interlocking of abstract and
concrete grammars, was found to be a very sound choice, but the solution
had several shortcommings that could not be addressed in that project.
For the present purposes, the three that worried us the most are the
following:
\begin{itemize}
\item
Recent developement of the \GF\ system made our grammars go out of sync.
\item
The library was monolithic, hampering its usability and maintenance.
\item
It included too few languages, especially as seen from an Europan
perspective.
\end{itemize}

The springboard for the present library was the need to properly solve
these problems, inasmuch as this was regarded as one of the most promissing
prerequisites for all further advanced developments in machine processing
of mathematical texts.  Thus the main tasks were:
\begin{itemize}
\item
To design a \emph{modular} mathematics library structured according to
the semantic standards (content dictionaries) of \openmath{}
(see \cite{OpenMath} and section \ref{sec:the_library} below).
\item
To code it in the much cleaner modern \GF\ release for the few languages
mentioned above, and
\item
To write new code for a few additional languages (Bulgarian, Finnish,
German, Romanian and Swedish).
\end{itemize}
The first two points amount to a tidying of the original \webalt{}
programming methods. The third point represents not merely an addition
of a few more languages, but a thourough testing of the methods and
procedures enforced in the preceding steps. This testing is important in
order to secure the rules for the inclusion of further languages and for
a controlled uniform extension of the available grammars.

At the end, we seek to have a modular and efficient library able
to deliver services for a multitude of natural languages to mathematical content providers.


\section{Some basic \GF\ notions}

For convenience, we include a few notions about the \GF\ 
system that will ease the considerations about MGL in the next section.
For a thourough reference, see \cite{Ranta11}.

\GF\ is a functional language for multilingual grammar applications where:
\begin{itemize}
	\item Productions are declared in \emph{abstract modules} whose types are derived from the Martin-L\"of type theory;
	\item Definitions of these abstract productions are given in a \emph{concrete module} specific to each of the targeted languages.
\end{itemize} 

This way, any \GF\ application begins by specifying its abstract syntax. This
syntax contains declarations of \emph{categories} (the \GF\  name for
types) and \emph{functions} (the \GF\ name for constructor signatures) and
has to capture the \emph{semantic structure} of the application domain.
For example, to let \Nat{} stand for the type of natural numbers and
\Prop{} for propositions about natural numbers, the \GF\ syntax is
\begin{lstlisting}
cat Nat, Prop ;    
\end{lstlisting}
The Peano inductive construction of the naturals can be expressed as follows:
\begin{lstlisting}
fun
   Zero : Nat ;
   Succ : Nat -> Nat ;
\end{lstlisting}
The signatures for `even number' and `prime number' can be captured with
\begin{lstlisting}
fun
   Even, Prime : Nat -> Prop ;	
\end{lstlisting}
So we can now express, for example: `the successor of the successor of zero is prime'.

Finally, we can abstract the logical `not', `and' and `or' as follows:
\begin{lstlisting}
fun
     Not : Prop -> Prop ;
     And, Or : Prop -> Prop -> Prop ;    
\end{lstlisting}



\section{The library} % (fold)
\label{sec:the_library}

\MGL\ is a collection of GF modules which are being developed in the context
of the \molto\ project \cite{molto}.
The categories used in these modules
are in correspondence with all possible combinations of \textbf{Variable}
and \textbf{Value} with the mathematical types
\textbf{Number}, \textbf{Set}, \textbf{Tensor} and \textbf{Function}.
Thus the category \texttt{VarNum} denotes a numeric
variable like $x$, while \texttt{ValSet} denotes an actual set like ``the
domain of the natural logarithm''.  The distinction between variables and
values allows us to type-check productions like lambda abstractions that
require a variable as the first argument.  Obviously variables can be
promoted to values when needed.

Other categories stand for propositions, geometric constructions and
indices.

The library is organised in a matrix-like form, with an horizontal axis
ranging over the targeted natural languages.  At the moment these are:
Bulgarian,
Catalan, English, Finnish, French, German, Italian, Romanian, Spanish and
Swedish.

The vertical axis is for complexity and contains, from bottom to top, three
layers:

\begin{enumerate}
	\item\emph{Ground}. It deals with literals, indices and variables.
	\item\emph{OpenMath}. It is modelled after the \openmath\ \emph{Content
Dictionaries} (\CD's). Each \CD{} defines a collection of
mathematical objects. This is a \emph{de facto} standard for mathematical
semantics and for each \emph{Content Dictionary}
there is a corresponding module in this layer.
The \CD's are collected by the
\emph{OpenMath consortium} \cite{OpenMath}.
	\item\emph{Operations}.
This layer takes care of simple mathematical exercises. These appear
in drilling materials and usually begin with directives such as
`Compute', `Find', `Prove', `Give an example of', etc.
\end{enumerate}

The following tree is an example belonging to the \emph{OpenMath} layer:
\begin{lstlisting}
mkProp
(lt_num
  (abs (plus (BaseValNum (Var2Num x) (Var2Num y))))
  (plus (BaseValNum (abs (Var2Num x)) (abs (Var2Num y)))))
\end{lstlisting}
When linearized with the Spanish concrete grammar, it yields
\begin{quote}
El valor absoluto de la suma  de x y de y es menor que la suma del valor
absoluto  de x y del valor absoluto de y
\end{quote}
\noindent Similarly, the tree
\begin{lstlisting}
DoSelectFromN
  (Var2Num y)
  (domain (inverse tanh))
  (mkProp
    (gt_num
    (At cosh (Var2Num y))
     pi))
\end{lstlisting}
gives, when linearized with the English concrete grammar:
\begin{quote}
Select y from the domain of the inverse of the hyperbolic tangent such that
the hyperbolic cosine of y is greater than pi.
\end{quote}

% section the_library (end)

\section{Conclusions and further work} % (fold)
\label{sec:conclusions}

After a first step in which the main concern was tidying and modularizing
the \webalt{} prototype for simple mathematics exercises for the languages
Catalan,
English,
%Finnish,
French,
Italian and
Spanish,
we have extended it, in a second step, with the languages
Bulgarian, Finnish,
German, Romanian and Sweedish.
In this note we have described the \GF\ library produced so far for
multilingual mathematical text processing, which we call
\MGL\ (Mathematics Grammar Library). We have also indicated how it
originated in the \webalt\ project, its relation to \GF, and its
present functionality.

Further work has three main lines:
\begin{itemize}
\item
Addition of new languages,
like Danish, Dutch, Norwegian, Polish, Portuguese, Russian, \ldots
This is a continuation of the first two steps referred to above and
our assessment is that it can be done reliably with the methods and
procedures established so far. To some extent, the library modules for a
new language can be generated automatically up to a point
from lexical items provided by natives in that language.
\item
Describing a controlled procedure for the uniform and reliable extension
of the grammars according to new semantic needs. This is an important step
that is being researched from several angles. One important point is to
ascertain when a piece of mathematical text requires functionalities
(categories, constructors, operations) not covered yet.
\item
Advancing in the use of \MGL\ for the production of ever more
sophisticated artificial mathematics assistants. This is also the focus of
current research that includes a collaboration with statistical machine
translation methods, as in principle they can suggest grammatical structures
out of a corpus of mathematical sentences.
\end{itemize}

% section conclusions (end)


\nocite{*}
\bibliographystyle{eptcsstyle/eptcs}
%\bibliographystyle{eptcs}
\bibliography{webalt}
\end{document}