\documentclass[submission,copyright,creativecommons]{eptcs}
\providecommand{\event}{SOS 2007} % Name of the event you are submitting to
%\usepackage{breakurl}             % Not needed if you use pdflatex only.
\usepackage{listings}
\usepackage{color}
\definecolor{light-gray}{gray}{0.8}

\title{The GF Mathematics Library}
\author{Jordi Saludes
\institute{Universitat Polit\`ecnica de Catalunya}
\institute{Sistemes Avan\c cats de Control}
%Edifici GAIA, Terrassa (Spain)
\email{jordi.saludes@upc.edu}
\and
Sebastian Xamb\'o
\institute{Universitat Polit\`ecnica de Catalunya}
\institute{MA2, Edifici OMEGA, Barcelona (Spain)}
\email{\quad sebastia.xambo@upc.edu}
}
\def\titlerunning{MGL}
\def\authorrunning{J. Saludes \& S. Xamb\'o}
\begin{document}
\lstset{backgroundcolor=\color{light-gray},
		breaklines=true,
		tabsize=2} % language=GF
\maketitle


%%%%%%%%%%%%%%%%%%%%%%%
% Begin Notes SX
%%%%%%%%%%%%%%%%%%%%%%%

\newcommand{\molto}{\textsc{mOlto}}
\newcommand{\webalt}{\textsc{WebALT}}
\newcommand{\openmath}{\textsc{OpenMath}}
\newcommand{\CD}{\textsc{CD}}
\newcommand{\OM}{\textsc{OM}}
\newcommand{\MGL}{\textsc{MGL}}
\newcommand{\MMA}{\textsc{MMA}}
\newcommand{\CAS}{\textsc{CAS}}
\newcommand{\GF}{\textsc{GF}}

\newcommand{\Nat}{\texttt{Nat}}
\newcommand{\Prop}{\texttt{Prop}}

%\newcommand{\exclude}[1]{#1}
\newcommand{\exclude}[1]{}


\begin{abstract}
The aim of this paper is to present
the \emph{Mathematics Grammar Library}.
The main points are
the context in which it originated,
its current design and functionality
and the present development goals.
We also comment on possible future applications
in the area of artificial mathematics assistants.
\end{abstract}



%% This paragraph is meant to force citing of all items.
%% Change the df of \exclude above to omit this part.
\exclude{
Bib so far:
\cite{Caprotti_multilingualdelivery},
\cite{Caprotti06webalt!deliver},
\cite{Nganga06},
\cite{Caprotti05},
\cite{Strotmann05},
\cite{Strotmann05!ISAAC},
\cite{Carlson05},
\cite{OpenMath},
\cite{GF},
\cite{Mathbar},
\cite{E-LearningMathematics},
\cite{AutonomousLearners}
}




\section{Introduction}

An archetypal meeting point for natural language processing and mathematics
education is the realm of \emph{word problems}
(\cite{wikipedia-wordproblem, Verschaffel-Greer-DeCorte-2000}), a realm in
which mechanised mathematics assistants (\MMA) are expected to play an ever
more prominent role in the years to come.  The Mathematics Grammar Library
(\MGL) presented in this paper is not an \MMA, but our working hypothesis
is that it is a good basis on which to build useful \MMA's for learning and
teaching (cf. \cite{E-LearningMathematics, AutonomousLearners} for some
general clues on e-learning technologies).

In fact, we regard \MGL{} as an enabling technology for \emph{multilingual}
dialog systems capable of helping students in learning how to solve word
problems.  This assessment is based on the \MGL{} potential capabilities
for dealing effectively with a mixture of text and mathematical
expressions, and also for managing interactions with ancillary \CAS{}
systems.

To be more specific, the current general aim for MGL is to provide natural
language services for mathematical constructs at the level of high school
and college freshmen linear algebra and calculus.  At the present stage,
the concrete goal is to provide rendering of simple mathematical exercises
in multiple languages (see \cite{MathBar} for a demo).



\section{Origin}

For a closer view of \MGL, let us look briefly at its origins.  The
idea behind \MGL{} was born, to a good extend, on reflecting about one of
the key results of the \webalt{} project (see
\cite{Caprotti_multilingualdelivery, Caprotti06webalt!deliver}).  In
summary, the unfolding of this reflection went as follows.

One of the aims of \webalt{} was to produce a proof-of-concept
platform for the creation of a multilingual repository of \emph{simple}
mathematical problems with guaranteed quality of the (machine)
translations, in both linguistic and mathematical terms. The languages
envisioned were Catalan,
English,
Finnish,
French,
Italian and
Spanish.
Of these, Finnish, with its great complexities, could not be raised to
the same level of functionality as the others.

The WebALT prototype was successful and, as far as we know, that endeavour
brought about the first application of the GF system (\cite{GF,Ranta11})
for the multilingual translation of simple mathematical questions.  The
powerful \GF{} scheme, based on the perfect interlocking of abstract and
concrete grammars, was found to be a very sound choice, but the solution
had several shortcommings that could not be addressed in that project.
For the present purposes, the three that worried us the most are the
following:
\begin{itemize}
\item
The grammars did not work for later versions of \GF{} ($>$2.9).
\item
The library was not modular with respect to
semantic processing, and hence not easy to maintain.
\item
It included too few languages, especially as seen from an Europan
perspective.
\end{itemize}

The springboard for the present library was the need to properly solve
these problems, inasmuch as this was regarded as one of the most promissing
prerequisites for all further advanced developments in machine processing
of mathematical texts.  Thus the main tasks were:
\begin{itemize}
\item
To design a \emph{modular} mathematics library structured according to
the semantic standards (content dictionaries) of \openmath{}
(\cite{OpenMath}).
\item
To code it in the much more expressive \GF{}\,\,3.1 for the few languages
mentioned above, and
\item
To write new code for a few additional languages (Bulgarian, Finnish,
German, Romanian and Sweedish).
\end{itemize}
The first two points amount to a tidying of the original \webalt{}
programming methods. The third point represents not merely an addition
of a few more languages, but a thourough testing of the methods and
procedures enforced in the preceding steps. This testing is important in
order to secure the rules for the inclusion of further languages and for
a controlled uniform extension of the available grammars.



\section{Some basic \GF{} notions}

For convenience, we include a few notions about the \GF{}
system that will ease the considerations about MGL in the next section.
For a thourough reference, see \cite{Ranta11}.

Any GF application begins by specifying its \emph{abstract syntax}. This
syntax contains declarations of \emph{categories} (the \GF{} name for
types) and \emph{functions} (the \GF{} name for constructor signatures) and
has to capture the \emph{semantic structure} of the application domain.
For example, to let \Nat{} stand for the type of natural numbers and
\Prop{} for propositions about natural numbers, the \GF{} syntax is
{\small
\begin{verbatim}
     cat Nat, Prop ;
\end{verbatim}
}
That `zero is a natural number' and that `the successor of
any natural number is a natural number' can be expressed as follows:
{\small
\begin{verbatim}
     fun
        Zero : Nat ;
        Succ : Nat -> Nat ;
\end{verbatim}
}
The signatures for `even number' and `prime number' can be captured with
{\small
\begin{verbatim}
     fun
        Even, Prime : Nat -> Prop ;
\end{verbatim}
}
Finally, we can abstract the logical `not', `and' and `or' as follows:
{\small
\begin{verbatim}
     fun
        Not : Prop -> Prop ;
        And, Or : Prop -> Prop -> Prop ;
\end{verbatim}
}
In practical terms, these declarations would form the \emph{body} of an
\emph{abstract module} that would have the form
{\small
\begin{verbatim}
     abstract Arith = {<body>}
\end{verbatim}
}
where \texttt{Arith} is the name of the module.



\section{The library} % (fold)
\label{sec:the_library}

\MGL{} is a collection of GF modules. The categories used in these modules
are in correspondence with all possible combinations of \textbf{Variable}
and \textbf{Value} with the mathematical types
\textbf{Number}, \textbf{Set}, \textbf{Tensor} and \textbf{Function}.
Thus the category \texttt{VarNum} denotes a numeric
variable like $x$, while \texttt{ValSet} denotes an actual set like ``the
domain of the natural logarithm''.  The distinction between variables and
values allows us to type-check productions like lambda abstractions that
require a variable as the first argument.  Obviously variables can be
promoted to values when needed.

Other categories stand for propositions, geometric constructions and
indices.

The library is organised in a matrix-like form, with an horizontal axis
ranging over the targeted natural languages.  At the moment these are:
Bulgarian,
Catalan, English, Finnish, French, German, Italian, Romanian, Spanish and
Swedish.

The vertical axis is for complexity and contains from bottom to top, three
layers:

\begin{enumerate}
\item\emph{Ground}. It deals with literals, indices and variables.
\item\emph{OpenMath}. It is modelled after the \OM{ } \emph{Content
Dictionaries} (\CD's). Each \CD{} defines a collection of
mathematical objects. This is a \emph{de facto} standard for mathematical
semantics and for each \emph{Content Dictionary}
there is a corresponding module in this layer.
The \CD's are collected by the
\emph{OpenMath consortium} \cite{OpenMath}.
\item\emph{Operations}.
This layer takes care of simple mathematical exercises. These appear
in drilling materials and usually begin with directives such as
`Compute', `Find', `Prove', `Give an example of', \ldots
\end{enumerate}

The following tree is an example belonging to the \emph{OpenMath} layer:
\begin{lstlisting}
     mkProp
     (lt_num
       (abs (plus (BaseValNum (Var2Num x) (Var2Num y))))
       (plus (BaseValNum (abs (Var2Num x)) (abs (Var2Num y)))))
\end{lstlisting}
When linearized with the Spanish concrete grammar, it yields
\begin{quote}
El valor absoluto de la suma  de x y de y es menor que la suma del valor
absoluto  de x y del valor absoluto de y
\end{quote}


Similarly, the tree
\begin{lstlisting}
     DoSelectFromN
       (Var2Num y)
       (domain (inverse tanh))
       (mkProp
         (gt_num
         (At cosh (Var2Num y))
          pi))
\end{lstlisting}
gives, when linearized with the English concrete grammar:
\begin{quote}
Select y from the domain of the inverse of the hyperbolic tangent such that
the hyperbolic cosine of y is greater than pi.
\end{quote}

% section the_library (end)

%%%%%%%%%%
% un exemple d'Operations?

\section{Conclusions and further work}

After a first step in which the main concern was tidying and modularizing
the \webalt{} prototype for simple mathematics exercises for the languages
Catalan,
English,
%Finnish,
French,
Italian and
Spanish,
we have extended it, in a second step, with the languages
Bulgarian, Finnish,
German, Romanian and Sweedish
(Finnish was considered in the first step, but it had to be worked out from
scratch).
In this note we have described the \GF{} library produced so far for
multilingual mathematical text processing, which we call
\MGL{} (Mathematics Grammar Library). We have also indicated how it
originated in the \webalt{} projecte, its relation to \GF{}, and its
present functionality.

Further work has three main lines:
\begin{itemize}
\item
Addition of new languages,
like Danish, Dutch, Norwegian, Polish, Portuguese, Russian, \ldots
This is a continuation of the first two steps referred to above and
our assessment is that it can be done reliably with the methods and
procedures established so far. To some extent, the library modules for a
new language can be generated automatically up to a point in which the
remaining work corresponds to natives in that language.
\item
Describing a controlled procedure for the uniform and reliable extension
of the grammars according to new semantic needs. This is an important step
that is being researched from several angles. One important point is to
ascertain when a piece of mathematical text requires functionalities
(categories, constructors, operations) not covered yet.
\item
Advancing in the use of \MGL{} for the production of ever more
sophisticated artificial mathematics assistants. This is also the focus of
current research that includes a collaboration with statistical machine
translation methods, as in principle they can suggest grammatical structures
out of a corpus of mathematical sentences.
\end{itemize}




%%%%%%%%%%%%%%%%%%%%
%% Excluded text
%%%%%%%%%%%%%%%%%%%%
\exclude{
\subsubsection*{Acknowledgements}
We are grateful for the generous help of many people and
institutions: The \webalt{} and \molto{} projects

WebALT

MOLTO steering committee

Katerina Bohmov\'a, Alba Hierro, Ramona Enache,

Francesc Massan\'es, Thomas Hallgren, Inari Listenmaa, Krasimir Angelov

Barcelona MT team: Cristina Espa\~na-Bonet, Llu\'{\i}s M\'arquez, Xavier
Carreras
}


\exclude{
\textbf{word problem}: a type of textbook problem
designed to help students
apply abstract mathematical concepts
to `real-world' situations.
Significant background information on the
problem is presented as text rather than in
mathematical notation.

\begin{itemize}
\item[]
A farm raises two kinds of animals: ducks and rabbits. In
total there are 60 animals and 140 legs. How many ducks
and rabbits are there in the farm?
\item[]
John is David's father. He is three times as old as his
son, but in 10 years, he will be twice as old. How old are
they now?
\end{itemize}
}

\exclude{
Another motivating experience.
In 1991 A. Deledicq started in France the \emph{Kangourou sans
fronti\`eres}, a mathematical contest for secondary education students that
P. O'Holloran had been running in Australia for thousands of students.  In
the following years, it was introduced progressively, with similar names,
in many other countries around the world.  The contest has four levels,
according to age (13 to 16). Each level consists of thirty
questions, the allowed time is 75 minutes and calculators are not allowed.
At present it is taken by hundreds of thousands of students.

For our research, one interesting aspect is that it is inheretly a
multilingual undertaking.  The problems are proposed by all national
mathematical societies that have adopted the contest and are chosen by a
large international committee composed of delegates from those societies.
Once chosen, the problems are translated by each society, so that in
principle there exists a very large body of mathematical questions with
human translations into more than forty languages.  Unfortunately, the
national societies tend to adapt the questions to their curricula and this
causes that in many instances we do not have a completely faithful
translation.

Here are two examples that hopefully convey a flavour of the kind of
questions that are proposed.
{\small
\begin{itemize}
\item[]
(Proposed by Russia) What is the least number of digits you need to delete
in the number 12323314 in order to find a palindrome?

   A) 1 B) 2 C) 3 D) 4 E) 5

\item[]
(Proposed by Ukraine) We have three boxes: one is white, another is red and
the last is black. One is empty, one has an apple, and another has a chocolate
bar. Find what color is the box that has the chocolate bar if we know that
the chocolate is in the white box or in red box and the apple is neither in
the white box nor in the black box.

A) White B) Red C) Black D) Red or Black E) It is impossible to know
\end{itemize}
}

} % end exclude


\nocite{*}
\bibliographystyle{eptcsstyle/eptcs}
%\bibliographystyle{eptcs}
\bibliography{webalt}
\end{document}


